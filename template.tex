\documentclass[10pt,a4paper]{article}
%\usepackage{zh_CN-Adobefonts_external}
\usepackage{xeCJK}
\usepackage{amsmath, amsthm}
\usepackage{listings,xcolor}
\usepackage{geometry} % 设置页边距
\usepackage{fontspec}
\usepackage{graphicx}
\usepackage[colorlinks]{hyperref}
\usepackage{setspace}
\usepackage{fancyhdr} % 自定义页眉页脚


\setsansfont{Fira Code} % 设置英文字体
\setmonofont[Mapping={}]{Fira Code} % 英文引号之类的正常显示,相当于设置英文字体

\linespread{1.2}

\title{Template For ICPC}
\author{Beyond List @ SDNU}
\definecolor{dkgreen}{rgb}{0,0.6,0}
\definecolor{gray}{rgb}{0.5,0.5,0.5}
\definecolor{mauve}{rgb}{0.58,0,0.82}

\pagestyle{fancy}

\lhead{\CJKfamily{kai} Shandong Normal University} %以下分别为左中右的页眉和页脚
\chead{}

\rhead{\CJKfamily{kai} 第 \thepage 页}
\lfoot{} 
\cfoot{\thepage}
\rfoot{}
\renewcommand{\headrulewidth}{0.4pt} 
\renewcommand{\footrulewidth}{0.4pt}
%\geometry{left=2.5cm,right=3cm,top=2.5cm,bottom=2.5cm} % 页边距
\geometry{left=3.18cm,right=3.18cm,top=2.54cm,bottom=2.54cm}
\setlength{\columnsep}{30pt}

\makeatletter

\makeatother



\lstset{
    language    = c++,
    numbers     = left,
    numberstyle={                               % 设置行号格式
        \small
        \color{black}
        \fontspec{Fira Code}
    },
	commentstyle = \color[RGB]{0,128,0}\bfseries, %代码注释的颜色
	keywordstyle={                              % 设置关键字格式
        \color[RGB]{40,40,255}
        \fontspec{Fira Code Bold}
        \bfseries
    },
	stringstyle={                               % 设置字符串格式
        \color[RGB]{128,0,0}
        \fontspec{Fira Code}
        \bfseries
    },
	basicstyle={                                % 设置代码格式
        \fontspec{Fira Code}
        \small\ttfamily
    },
	emphstyle=\color[RGB]{112,64,160},          % 设置强调字格式
    breaklines=true,                            % 设置自动换行
    tabsize     = 4,
    frame       = single,%主题
    columns     = fullflexible,
    rulesepcolor = \color{red!20!green!20!blue!20}, %设置边框的颜色
    showstringspaces = false, %不显示代码字符串中间的空格标记
	escapeinside={\%-1367342336}{*)},
}

\begin{document}
\title{ICPC Templates}
\author {Beyond List @ SDNU}
\maketitle
\tableofcontents
\newpage
\section{Math}
\subsection{欧拉筛}
\lstinputlisting{Math/Sieve.cpp}
\subsection{组合数}
\lstinputlisting{Math/Comb.cpp}
\subsection{拓展欧几里得}
\begin{spacing}{1.5}
对于方程 $ax + by + c$, 调用$`exgcd`$, 求出 $x_0$ 和 $y_0$,
使得 $ax_0 + by_0 = gcd(a, c)$。

在 $gcd(a, b) | c$的情况下,方程有通解:

$x = x_0 * \frac{c}{gcd(a, b)} + k * \frac{b}{gcd(a, b)}$ 和
$y = y_0 * \frac{c}{gcd(a, b)} - k * \frac{a}{gcd(a, b)}$

\end{spacing}
\lstinputlisting{Math/ExGcd.cpp}
\subsection{中国剩余定理}
\lstinputlisting{Math/ExCrt.cpp}
\subsection{RandomTheory}
\subsubsection{RandomNumber}
\lstinputlisting{Math/RandomTheory/Random.cpp}
\subsubsection{MillerRabin}
\lstinputlisting{Math/RandomTheory/MillerRabin.cpp}
\subsubsection{PollardRho}
\begin{spacing}{1.5}
如果 $n$ 是质数(MillerRabin判断), 返回 $n$, 否则返回 $n$ 
的随机一个 $[2, n - 1]$的因子。

复杂度略微高于$O(n^{\frac{1}{4}}logn)$

\end{spacing}
\lstinputlisting{Math/RandomTheory/PollardRho.cpp}
\section{String}
\subsection{最小表示法}
\lstinputlisting{String/MinimalString.cpp}
\subsection{字符串哈希}
\lstinputlisting{String/StringHash.cpp}
\subsection{KMP}
\lstinputlisting{String/KMP.cpp}
\subsection{字典树}
\lstinputlisting{String/Trie.cpp}
\subsection{01字典树}
\lstinputlisting{String/01Trie.cpp}
\subsection{AC自动机}
\lstinputlisting{String/ACAutomaton.cpp}
\subsection{AC自动机2}
\lstinputlisting{String/ACAutomaton2.cpp}
\subsection{Z-Function}
\begin{spacing}{1.5}
$Z_i$ 是 $S$ 与 $S[i\dots n - 1]$ 的最长公共前缀
\end{spacing}
\lstinputlisting{String/zFunction.cpp}
\subsection{马拉车}
\lstinputlisting{String/Manacher.cpp}
\subsection{后缀数组}
\begin{spacing}{1.5}
字符串本质不同的子串数量为 $\frac{n(n - 1)}{2} - \sum h[i]$

两个子串的 LCP 为 $\min_{l_1 \le k \le l_2}h[k]$
\end{spacing}
\lstinputlisting{String/SuffixArray.cpp}
\section{DataStruct}
\subsection{RMQ}
\lstinputlisting{DataStruct/RMQ.cpp}
\subsection{Heap}
\lstinputlisting{DataStruct/Heap.cpp}
\subsection{并查集}
\lstinputlisting{DataStruct/DSU.cpp}
\subsection{带权并查集}
\lstinputlisting{DataStruct/WeightDSU.cpp}
\subsection{pbdsTree}
\lstinputlisting{DataStruct/pbdsTree.cpp}
\subsection{SegmentTree}
\subsubsection{SegTree}
\lstinputlisting{DataStruct/SegmentTree/SegTree.cpp}
\subsubsection{LazySegTree}
\lstinputlisting{DataStruct/SegmentTree/LazySegTree.cpp}
\subsubsection{主席树}
\lstinputlisting{DataStruct/SegmentTree/PresidentTree.cpp}
\subsection{BIT}
\subsubsection{BIT}
\lstinputlisting{DataStruct/BIT/BIT.cpp}
\subsubsection{RangeBIT}
\lstinputlisting{DataStruct/BIT/RangeBIT.cpp}
\subsubsection{MatBIT}
\lstinputlisting{DataStruct/BIT/MatBIT.cpp}
\subsubsection{RangeMatBIT}
\lstinputlisting{DataStruct/BIT/RangeMatBIT.cpp}
\subsection{Block}
\lstinputlisting{DataStruct/Block.cpp}
\section{Graph}
\subsection{树剖}
\lstinputlisting{Graph/TreePre.cpp}
\subsection{树的直径}
\lstinputlisting{Graph/TreeDiameter.cpp}
\subsection{树哈希}
\lstinputlisting{Graph/TreeHash.cpp}
\subsection{最短路}
\lstinputlisting{Graph/Dijkstra.cpp}
\subsection{二分图染色}
\lstinputlisting{Graph/Bi-GraphColor.cpp}
\subsection{拓扑排序}
\lstinputlisting{Graph/TopSort.cpp}
\subsection{连通性}
\subsubsection{强连通分量}
\lstinputlisting{Graph/连通性/StronglyConnectedComponent.cpp}
\subsubsection{割点}
\lstinputlisting{Graph/连通性/VertexBiconnectedComponent.cpp}
\subsubsection{割边}
\lstinputlisting{Graph/连通性/EdgeBiconnectedComponent.cpp}
\subsection{网络流}
\subsubsection{最大流}
\lstinputlisting{Graph/网络流/MaxFlow.cpp}
\subsubsection{最小费用流}
\lstinputlisting{Graph/网络流/MinCostFlow.cpp}
\section{Optimize}
\subsection{快读快写}
\lstinputlisting{Optimize/fastIO.cpp}
\subsection{手写哈希}
\lstinputlisting{Optimize/Hash.cpp}
\end{document}